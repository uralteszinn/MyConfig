::scrarticle::
\documentclass[paper=a4,pagesize,11pt,draft=false]{scrartcl}
<<VIM:EMPTY>>
\usepackage[utf8]{inputenc}
\usepackage[english]{babel}
\usepackage{scrpage2}               
%insertUsepackages
\usepackage{hyperref}
%=== Mine =====
\usepackage{mycommands}
<<VIM:EMPTY>>
%===================================================
<<VIM:EMPTY>>
\setlength{\parindent}{0ex}
<<VIM:EMPTY>>
%===================================================
<<VIM:EMPTY>>
%insertNTheorem
<<VIM:EMPTY>>
%===================================================
<<VIM:EMPTY>>
\clearscrheadings
\clearscrplain
<<VIM:EMPTY>>
\ihead[]{}
\chead[<<VIM:INPUT|1|Titel>>]{<<VIM:INPUT|1>>}
\ohead[]{}
<<VIM:EMPTY>>
\ifoot[<<VIM:INPUT|2|Autor>>]{<<VIM:INPUT|2>>}
\cfoot[]{}
\ofoot[<<VIM:OPTIONINPUT|1|Pagemark Choice|\pagemark,|||Pagemark,None>>]{<<VIM:OPTIONINPUT|1>>}
<<VIM:EMPTY>>
\pagestyle{scrheadings}
<<VIM:EMPTY>>
%===================================================
<<VIM:EMPTY>>
\begin{document}
<<VIM:EMPTY>>
%insertTitle
<<VIM:EMPTY>>
%\tableofcontents
<<VIM:EMPTY>>
<++>
<<VIM:EMPTY>>
\end{document}
<<VIM:EMPTY>>
%insertVim


::scrletter::
\documentclass[paper=a4,pagesize,11pt,draft=false]{scrlttr2}
<<VIM:EMPTY>>
\usepackage[utf8]{inputenc}
\usepackage[ngerman]{babel}
%\usepackage{scrpage2}
%insertUsepackages
\usepackage{hyperref}
<<VIM:EMPTY>>
%===================================================
<<VIM:EMPTY>>
\setlength{\parindent}{0ex}
<<VIM:EMPTY>>
%===================================================
<<VIM:EMPTY>>
%insertNTheorem
<<VIM:EMPTY>>
%===================================================
<<VIM:EMPTY>>
\LoadLetterOption{SN}
<<VIM:EMPTY>>
\setkomavar{fromname}{\ttfamily Laurin Stenz}
\setkomavar{fromaddress}{\ttfamily Am Bogen 1\\ 5620 Bremgarten}
\setkomavar{fromphone}{\ttfamily Laurin Stenz\\\ttfamily 079\hspace{0.3ex}/\hspace{0.3ex}916 90 47}
\setkomavar{fromemail}{\url{laurinstenz@gmail.com}}
\setkomavar*{fromphone}{}
\setkomavar*{fromemail}{}
\setkomavar{place}{Bremgarten}
\setkomavar{title}{<<VIM:INPUT|5|Title>>}
\setkomavar{subject}{<<VIM:INPUT|6|Subject>>}
\setkomavar{signature}{Laurin Stenz}
\KOMAoptions{fromemail,fromphone,fromalign=left,foldmarks=false}
<<VIM:EMPTY>>
%===================================================
<<VIM:EMPTY>>
\begin{document}
<<VIM:EMPTY>>
\begin{letter}{<<VIM:INPUT|1|Addresser>>\\<<VIM:INPUT|2|Street>>\\<<VIM:INPUT|3|PLZ>> <<VIM:INPUT|4|Place>>}
    \opening{<<VIM:INPUT|7|Opening>>}
        <++>
    \closing{<<VIM:INPUT|8|Closing>>\\[5mm]}
\end{letter}
<<VIM:EMPTY>>
\end{document}
<<VIM:EMPTY>>
%insertVim

::beamer::
\documentclass[hyperref={pdfpagelabels=false}]{beamer}
<<VIM:EMPTY>>
\usepackage[english]{babel}
\usepackage[utf8]{inputenc}
\usepackage{scrpage2}
%insertUsepackages
\usepackage{hyperref}
%=== Mine =====
\usepackage{mycommands}
<<VIM:EMPTY>>
%===================================================
<<VIM:EMPTY>>
\setlength{\parindent}{0ex}
<<VIM:EMPTY>>
%===================================================
<<VIM:EMPTY>>
%insertNTheorem
<<VIM:EMPTY>>
%===================================================
<<VIM:EMPTY>>
\title{<<VIM:INPUT|1|Title Presentation>>}   
\author{<<VIM:INPUT|2|Author>>} 
\date{<<VIM:INPUT|3|Date>>} 
<<VIM:EMPTY>>
%===================================================
<<VIM:EMPTY>>
\usetheme{Hannover}
<<VIM:EMPTY>>
\begin{document}
<<VIM:EMPTY>>
\begin{frame}
    \titlepage
\end{frame} 
<<VIM:EMPTY>>
\begin{frame}
    \frametitle{Table of Contents}
    \tableofcontents
\end{frame} 
<<VIM:EMPTY>>
\begin{frame}
    \frametitle{<<VIM:INPUT|4|Title First Slide>>}
    \framesubtitle{<<VIM:INPUT|5|Subtitle First Slide>>}
    <++>
\end{frame}<++>
<<VIM:EMPTY>>
\end{document}
<<VIM:EMPTY>>
%insertVim

::insertVim::
% VIM: CompileDirectory = <<VIM:DIRPATH>>
% VIM: CompileCommand = pdflatex '<<VIM:FILEPATH>>'
% VIM: ViewCommand = evince '<<VIM:DIRPATH>>/<<VIM:FILENAMENOEXTENSION>>.pdf' &<++>

::insertUsepackages::
%\usepackage{scrdate}
%\usepackage{scrtime}
%\usepackage{lastpage}
\usepackage{tikz}
%\usepackage{setspace}
\usepackage{array}
\usepackage{multirow}
%\usepackage{enumitem}
%\usepackage{enumerate}
%\usepackage{color}
\usepackage{xcolor}
\usepackage{graphicx}
\usepackage[intlimits]{amsmath}
\usepackage{amssymb}
\usepackage{mathtools}
%\usepackage{extpfeil}
%\usepackage{marvosym}
%\usepackage{wasysym}
%\usepackage{pifont}
\usepackage[framed,thmmarks,thref,amsmath]{ntheorem}
%\usepackage[ntheorem]{empheq}
\usepackage{framed}<++>


::insertTitle::
%\subject{}
\title{<<VIM:INPUT|1|Title>>}
%\subtitle{}
\author{<<VIM:INPUT|2|Author>>}
\date{<<VIM:INPUT|3|Date>>}
%\publishers{}
%\dedication{}
%\thanks{}
%\uppertitleback{}
%\lowertitleback{}
%\extratitle{}
%\titlehead{}
<<VIM:EMPTY>>
\maketitle<++>




::insertNTheorem::
\usepackage{mythm}
<<VIM:EMPTY>>
%\renewcommand{\mytheoremheaderfont}{\color{blue}\normalfont\bfseries}
%\renewcommand{\mytheoremoptheaderfont}{\normalfont\scriptsize}
%\theorembodyfont{\normalfont}
<<VIM:EMPTY>>
%\theoremindent0.5ex
%\setlength{\theorempreskipamount}{0.5ex}
%\setlength{\theorempostskipamount}{0.5ex}
<<VIM:EMPTY>>
%\definecolor{gray}{gray}{0.8}
%\def\theoremframecommand{\colorbox{gray}}
<<VIM:EMPTY>>
\theoremstyle{mynonumberthm}
\newframedtheorem{defn}{Definition}[section]
<<VIM:EMPTY>>
\theoremstyle{mythm}
\newframedtheorem{thm}{Theorem}[section]
\newframedtheorem{satz}[thm]{Satz}
\newframedtheorem{prop}[thm]{Proposition}
\newframedtheorem{lem}[thm]{Lemma}
\newframedtheorem{cor}[thm]{Corollary}
\newframedtheorem{kor}[thm]{Korollar}
<<VIM:EMPTY>>
\theoremstyle{mynonumberthm}
\theoremindent0ex
\theoremsymbol{{\color{black}/\!\!/}}
\newtheorem{ex}{Example}[section]
\newtheorem{rem}{Remark}[section]
\newtheorem{bsp}{Beispiel}[section]
\newtheorem{bem}{Bemerkung}[section]
<<VIM:EMPTY>>
\theoremstyle{mybew}
\theoremindent0ex
\theoremsymbol{{\color{black}\(\blacksquare\)}}
\newtheorem{pf}{Proof}
\newtheorem{bew}{Beweis}<++>



::table::
\begin{table}<<VIM:INPUT|1|Optional Argument||[|]>>
    \centering
    <<VIM:EMPTY>>
    \begin{tabular}{<++>}
        <++>
    \end{tabular}
    <<VIM:EMPTY>>
    \caption{<<VIM:INPUT|2|Name>>}
    \label{table-<<VIM:RANDOM>>}
\end{table}<++>


::tabular::
\begin{tabular}{<++>}
    <++>
\end{tabular}<++>

::multicolumn::
\multicolumn{<++>}{<++>}{<++>}<++>

::multirow::
\multirow{<++>}{<++>}{<++>}<++>

::itemize::
\begin{itemize}
<<VIM:TAB>>\item <++>
\end{itemize}

::enumerate::
\begin{enumerate}
<<VIM:TAB>>\item <++>
\end{enumerate}

::description::
\begin{description}
<<VIM:TAB>>\item[<++>] <++>
\end{description}

::item::
<<VIM:EMPTY>>
<<VIM:NOBEFOREPART>>\item<<VIM:INPUT|1|Optional Argument||[|]>> <++>

::footnote::
\footnote{<<VIM:CURSOR}<++>



::part::
\part<<VIM:CHOICE|1|With Number?|,*>>{<<VIM:INPUT|1|Title>>}
\label{part-<<VIM:RANDOM>>}
<++>

::chapter::
\chapter<<VIM:CHOICE|1|With Number?|,*>>{<<VIM:INPUT|1|Title>>}
\label{chapter-<<VIM:RANDOM>>}
<++>

::section::
\section<<VIM:CHOICE|1|With Number?|,*>>{<<VIM:INPUT|1|Title>>}
\label{section-<<VIM:RANDOM>>}
<++>

::subsection::
\subsection<<VIM:CHOICE|1|With Number?|,*>>{<<VIM:INPUT|1|Title>>}
\label{subsection-<<VIM:RANDOM>>}
<++>

::subsubsection::
\subsubsection<<VIM:CHOICE|1|With Number?|,*>>{<<VIM:INPUT|1|Title>>}
\label{subsubsection-<<VIM:RANDOM>>}
<++>

::paragraph::
\paragraph<<VIM:CHOICE|1|With Number?|,*>>{<<VIM:INPUT|1|Title>>}
\label{paragraph-<<VIM:RANDOM>>}
<++>

::subparagraph::
\subparagraph<<VIM:CHOICE|1|With Number?|,*>>{<<VIM:INPUT|1|Title>>}
\label{subparagraph-<<VIM:RANDOM>>}
<++>

::label::
\label{<<VIM:INPUT|1|Category>>-<<VIM:RANDOM>>}<++>

::figure::
\begin{figure}<<VIM:INPUT|1|Optional Argument||[|]>>
    \centering
    <<VIM:EMPTY>>
    <++>
    <<VIM:EMPTY>>
    \caption{<<VIM:INPUT|1|Name>>}
    \label{figure-<<VIM:RANDOM>>}
\end{figure}<++>

::img::
\includegraphics<<VIM:INPUT|1|Optional Height||[height=|]>>{<++>}<++>

::frame::
\begin{frame}
    \frametitle{<<VIM:INPUT|4|Title>>}
    \framesubtitle{<<VIM:INPUT|5|Subtitle>>}
    <++>
\end{frame}<++>


::lim::
\lim<++>_{<++>}<++>

::int::
\<++>int_{<++>}^{<++>}<++>\;\mathrm{d}<++>

::sum::
\sum_{<++>}^{<++>}<++>

::prod::
\prod_{<++>}^{<++>}<++>

::frac::
\frac{<++>}{<++>}<++>

::sqrt::
\sqrt[<++>]{<++>}<++>

::bigcup::
\bigcup_{<++>}^{<++>}<++>

::bigcap::
\bigcap_{<++>}^{<++>}<++>

::bigoplus::
\bigoplus_{<++>}^{<++>}<++>

::bigominus::
\bigominus_{<++>}^{<++>}<++>

::bigodot::
\bigodot_{<++>}^{<++>}<++>

::bigotimes::
\bigotimes_{<++>}^{<++>}<++>

::d::
\mathrm{d}<++>

::rm::
\mathrm{<++>}<++>

::bb::
\mathbb{<++>}<++>

::cal::
\mathcal{<++>}<++>

::t::
\text{<++>}<++>

::diff::
\frac{\mathrm{d<++>}}{\mathrm{d}<++>}<++>

::R::
\mathbb{R}<++>

::pmatrixinline::
\begin{pmatrix} <++> \end{pmatrix} <++>

::smallmatrix::
\left( \begin{smallmatrix} <++> \end{smallmatrix} \right) <++>

::tikz::
\begin{tikzpicture}<<VIM:INPUT|1|Optional Argument||[|]>>
    <++>
\end{tikzpicture}

::coordinate::
\coordinate (<++>) at (<++>);<++>

::path::
\path<<VIM:INPUT|1|Optional Argument||[|]>> <++>;<++>

::draw::
\draw<<VIM:INPUT|1|Optional Argument||[|]>> <++>;<++>

::fill::
\fill<<VIM:INPUT|1|Optional Argument||[|]>> <++>;<++>

::filldraw::
\filldraw<<VIM:INPUT|1|Optional Argument||[|]>> <++>;<++>

::pt::
<++>(<++>)<++>

::node::
\node<<VIM:INPUT|1|Optional Argument||[|]>><<VIM:INPUT|2|Name|| (|)|>> at (<++>) {<++>};<++>

::circle::
<++>(<++>) circle (<++>)<++>

::rectangle::
(<++>) rectangle <++>(<++>)<++>

::foreach::
\foreach <++> in {<++>} {
<<VIM:TAB>><++>
}

::defaultName::
\begin{<<VIM:NAME>>}
<<VIM:TAB>><++>
\end{<<VIM:NAME>>}<++>

::defaultNoName::
\begin{align*}
<<VIM:TAB>><++>
\end{align*}<++>

::test::
<<VIM:OPTION|1|Gugus|a,b,c,d,f,g|>>
<<VIM:OPTIONINPUT|2|Gogos|a,b,c,d,f,g|>>
